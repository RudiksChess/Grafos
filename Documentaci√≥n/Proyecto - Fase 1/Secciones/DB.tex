\section{Diseño de la base de datos}
Un resumen de la base de datos original obtenida es la presentada en la parte inferior. Su implementación se basa en relacionarla con grafos. El propósito: 

\begin{itemize}
    \item \textbf{Nodos} - Los valores de las columnas.
    \item \textbf{Relaciones} - Los títulos de las columnas, exceptuando «USER». 
\end{itemize}
\begin{table}[ht]
\resizebox{\columnwidth}{!}{%
\begin{tabular}{@{}lllllll@{}}
\toprule
USER    & NIVEL\_BLITZ & NIVEL\_RAPIDAS & PARTE\_FAVORITA & PLATAFORMA  & APERTURA          & DEFENSA   \\ \midrule
user001 & Principiante & Intermedio     & Intermedio      & Chess.com   & Inglesa           & Siciliana \\
user002 & Principiante & Principiante   & Final           & Chess.com   & Fianchetto        & Siciliana \\
user003 & Principiante & Principiante   & Final           & Chess.com   & Italiana/Española & Francesa  \\
user004 & Principiante & Principiante   & Final           & Chess.com   & Italiana/Española & Francesa  \\
user005 & Principiante & Principiante   & Intermedio      & Chess.com   & Italiana/Española & Francesa  \\
user006 & Principiante & Principiante   & Intermedio      & Chess.com   & Italiana/Española & Eslava    \\
user007 & Avanzado     & Avanzado       & Intermedio      & Chess.com   & Sistema Londres   & Caro-Kann \\
user008 & Principiante & Principiante   & Final           & Lichess.org & Fianchetto        & Caro-Kann \\
user009 & Intermedio   & Intermedio     & Apertura        & Chess.com   & Italiana/Española & Siciliana \\
user010 & Principiante & Principiante   & Final           & Chess.com   & Italiana/Española & Francesa  \\ \bottomrule
\end{tabular}%
}
\end{table}

\section{Repositorio}

El repositorio del proyecto se encuentra en el siguiente link:\newline 
\textcolor{blue}{\href{https://github.com/RudiksChess/Grafos}{https://github.com/RudiksChess/Grafos}}


